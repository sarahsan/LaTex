\documentclass[a4paper]{article}
\usepackage[latin1]{inputenc}
\usepackage[T1]{fontenc}
\usepackage[francais]{babel}
\usepackage[colorlinks=true]{hyperref} 
\hypersetup{urlcolor=blue,linkcolor=black,citecolor=black,colorlinks=true}

\title{Les R\'{e}seaux Mobiles}
\author{Achab Wissem , AitAmeur Lydia , Benhadj Djilali Hadjer ,Maatouk Asma}
\date{2013 - 2014}

\begin{document}

\maketitle
\newpage 
\renewcommand{\contentsname}{Sommaire}
\tableofcontents


\section{Introduction}
\section{Usages}
\section{R\'{e}seaux mobiles}
\section{Caract\'{e}ristiques}
\section{Avantages}
\section{Les diff\'{e}rents protocoles (IPv6)}
\subsection{Dualit\'{e} d'Identification / Localisation de l'adresse IP}
\subsection{Types de mobilit\'{e} et de Handovers}
\subsection{Solutions de la gestion de la mobilit\'{e}}
\subsection{Mobile IPv6 - Support de mobilit\'{e} pour IPv6 (MIPv6, RFC 3775,2004)}
\subsection{Fonctionnement du protocole MIPv6}
\subsection{Limites de performance de MIPv6 }
\section{Protocole NEMO}
\section{Contraintes}
\section{Conclusion}


\end{document}
